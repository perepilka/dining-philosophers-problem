\documentclass[12pt,a4paper]{article}

% Pakiety
\usepackage[T1]{fontenc}
\usepackage[utf8]{inputenc}
\usepackage{geometry}
\usepackage{graphicx}
\usepackage{amsmath}
\usepackage{hyperref}
\usepackage{xcolor}

% Polish characters support
\usepackage[polish]{babel}

% Ustawienia strony
\geometry{
    left=2.5cm,
    right=2.5cm,
    top=2.5cm,
    bottom=2.5cm
}

% Kolory
\definecolor{titlecolor}{RGB}{0,51,102}
\definecolor{subtitlecolor}{RGB}{102,102,102}

\begin{document}

% Strona tytułowa
\begin{titlepage}
    \centering
    
    % Logo/Nazwa uczelni (opcjonalnie)
    \vspace{2cm}
    
    {\Large\textsc{Politechnika Wrocławska}}\\[0.5cm]
    {\large\textsc{Wydział Informatyki i Telekomunikacji}}\\[2cm]
    
    % Przedmiot
    {\Large\textbf{Systemy Operacyjne}}\\[0.5cm]
    
    \rule{\textwidth}{1.5pt}\\[0.3cm]
    
    % Tytuł projektu
    {\Huge\color{titlecolor}\textbf{Problem Ucztujących}}\\[0.2cm]
    {\Huge\color{titlecolor}\textbf{Filozofów}}\\[0.3cm]
    {\Large\color{subtitlecolor}\textit{Dining Philosophers Problem}}\\[0.3cm]
    
    \rule{\textwidth}{1.5pt}\\[1.5cm]
    
    % Podtytuł
    {\large Implementacja i analiza porównawcza trzech strategii}\\
    {\large synchronizacji wielowątkowej}\\[3cm]
    
    % Informacje o autorze i prowadzącym
    \begin{minipage}{0.45\textwidth}
        \begin{flushleft}
            \textbf{Autor:}\\
            Yaroslav Perepilka
        \end{flushleft}
    \end{minipage}
    \hfill
    \begin{minipage}{0.45\textwidth}
        \begin{flushright}
            \textbf{Prowadzący:}\\
            dr inż. Mariusz Makuchowski
        \end{flushright}
    \end{minipage}
    
    \vfill
    
    % Data
    {\large Wrocław, 6 stycznia 2025}
    
\end{titlepage}

% Opcjonalnie: Spis treści
\newpage
\tableofcontents

\newpage

% Tu będzie treść sprawozdania
\section{Wstęp}

Problem ucztujących filozofów (ang. \textit{Dining Philosophers Problem}) jest klasycznym problemem synchronizacji procesów, sformułowanym przez Edsgara Dijkstrę w 1965 roku. Problem ten ilustruje wyzwania związane z zapobieganiem zakleszczeniom (\textit{deadlock}) i zagłodzeniu (\textit{starvation}) w systemach wielowątkowych.

\subsection{Opis problemu}

Pięciu filozofów siedzi przy okrągłym stole. Każdy filozof na przemian myśli i je. Do jedzenia potrzebne są dwa widelce -- lewy i prawy. Problem polega na zaprojektowaniu takiego protokołu, który:

\begin{itemize}
    \item Zapobiega zakleszczeniom (deadlock)
    \item Zapobiega zagłodzeniu pojedynczych filozofów (starvation)
    \item Maksymalizuje współbieżność (concurrency)
    \item Zapewnia sprawiedliwy dostęp do zasobów (fairness)
\end{itemize}

\subsection{Cel projektu}

Celem niniejszego projektu jest:

\begin{enumerate}
    \item Implementacja trzech różnych strategii synchronizacji
    \item Analiza porównawcza wydajności i sprawiedliwości
    \item Badanie wpływu liczby filozofów na zachowanie systemu
    \item Wykrycie i demonstracja sytuacji zakleszczenia
\end{enumerate}

\section{Implementacja}

\subsection{Środowisko techniczne}

Projekt został zrealizowany w języku Python 3 z wykorzystaniem biblioteki standardowej \texttt{threading}. Wybór języka Python pozwala na czytelną implementację oraz łatwą analizę wyników.

\subsubsection{Specyfikacja sprzętowa}

Testy wydajnościowe przeprowadzono na następującym sprzęcie:

\begin{itemize}
    \item \textbf{Procesor:} AMD Ryzen 7 6800H with Radeon Graphics
    \item \textbf{Liczba rdzeni fizycznych:} 8
    \item \textbf{Liczba wątków (z SMT):} 16
    \item \textbf{System operacyjny:} Linux
\end{itemize}

\subsubsection{Wpływ wielordzeniowości na problem}

Liczba dostępnych rdzeni procesora ma kluczowe znaczenie dla problemu ucztujących filozofów:

\begin{enumerate}
    \item \textbf{Prawdziwa współbieżność:} Przy 16 dostępnych wątkach sprzętowych możliwa jest rzeczywista równoległa praca wielu filozofów jednocześnie (w przeciwieństwie do pseudo-równoległości na jednordzeniowych systemach).
    
    \item \textbf{Intensyfikacja race conditions:} Większa liczba rdzeni zwiększa prawdopodobieństwo wystąpienia rzeczywistych wyścigów (race conditions) i zakleszczenia, co lepiej pokazuje problemy synchronizacji.
    
    \item \textbf{Skalowalność:} Przy testach z 32 filozofami na 16 wątkach sprzętowych widoczne staje się znaczenie efektywnej synchronizacji -- w danym momencie jedynie połowa filozofów może wykonywać się równolegle.
    
    \item \textbf{Overhead context switching:} Gdy liczba filozofów przekracza liczbę wątków sprzętowych, scheduler systemu operacyjnego musi przełączać konteksty, co wpływa na wydajność.
\end{enumerate}

W przypadku strategii \textbf{deadlock}, wielordzeniowość przyspiesza wystąpienie zakleszczenia -- wszystkie wątki mogą niemal jednocześnie zająć lewe widelce, prowadząc do natychmiastowego deadlocka.

Dla strategii \textbf{hierarchy} i \textbf{asymmetric}, większa liczba rdzeni pozwala na wyższy throughput (więcej posiłków na sekundę), gdyż więcej filozofów może jeść równolegle bez blokowania się nawzajem.

\subsection{Zaimplementowane strategie}

\subsubsection{Strategia 1: Deadlock (Podejście naiwne)}

Wszyscy filozofowie działają jednolicie -- najpierw podnoszą lewy widelec, następnie prawy. Ta strategia prowadzi do zakleszczenia, gdy wszyscy filozofowie jednocześnie podniosą lewy widelec i czekają na prawy.

\subsubsection{Strategia 2: Hierarchy (Rozwiązanie Dijkstry)}

Widelce numerowane są od 0 do N-1. Każdy filozof zawsze podnosi widelec o niższym numerze jako pierwszy. To rozwiązanie przerywa cykliczne oczekiwanie (\textit{circular wait}), eliminując możliwość zakleszczenia.

\subsubsection{Strategia 3: Asymmetric (Filozofowie parzyści/nieparzyści)}

Filozofowie o parzystych indeksach podnoszą najpierw lewy widelec, a o nieparzystych -- prawy. Przerwanie symetrii działań zapobiega jednoczesnej próbie zajęcia tych samych zasobów.

\section{Metodologia badań}

\subsection{Parametry testów}

Przeprowadzono testy dla następujących konfiguracji:

\begin{itemize}
    \item \textbf{Liczba filozofów:} 3, 5, 16, 32
    \item \textbf{Czas symulacji:} 30s, 60s, 180s (3 min), 600s (10 min), 1200s (20 min)
    \item \textbf{Strategie:} deadlock, hierarchy, asymmetric
\end{itemize}

\subsection{Mierzone metryki}

\begin{itemize}
    \item Całkowita liczba posiłków (\textit{total meals})
    \item Średnia liczba posiłków na filozofa
    \item Przepustowość systemu (posiłki/sekundę)
    \item Współczynnik sprawiedliwości (\textit{fairness coefficient})
    \item Odchylenie standardowe liczby posiłków
    \item Wykrycie zakleszczenia
\end{itemize}

\section{Wyniki}

\subsection{Strategia Deadlock -- demonstracja zakleszczenia}

Strategia deadlock, jak oczekiwano, prowadzi do całkowitego zakleszczenia systemu. Wszystkie testy zakończyły się deadlockiem -- żaden filozof nie zdołał zjeść ani jednego posiłku.

\begin{table}[h]
\centering
\caption{Wyniki strategii Deadlock}
\begin{tabular}{|c|c|c|c|}
\hline
\textbf{Liczba filozofów} & \textbf{Czas [s]} & \textbf{Posiłki} & \textbf{Zakleszczeni} \\
\hline
3 & 1200 & 0 & 3 \\
5 & 1200 & 0 & 5 \\
16 & 1200 & 0 & 16 \\
32 & 1200 & 0 & 32 \\
\hline
\end{tabular}
\end{table}

\textbf{Obserwacje:}
\begin{itemize}
    \item Deadlock występuje niemal natychmiast (w pierwszych sekundach symulacji)
    \item Wszyscy filozofowie zostają zablokowany z jednym widelcem w ręku
    \item Niezależnie od liczby filozofów, wynik jest identyczny -- zero posiłków
    \item System nie może wyjść z deadlocka samoistnie
\end{itemize}

\subsection{Strategia Hierarchy -- rozwiązanie Dijkstry}

Strategia hierarchy całkowicie eliminuje możliwość deadlocka poprzez wprowadzenie hierarchii zasobów. Wszyscy filozofowie zawsze pobierają widelec o niższym numerze jako pierwszy.

\begin{table}[h]
\centering
\caption{Wyniki strategii Hierarchy -- pełne wyniki}
\small
\begin{tabular}{|c|c|c|c|c|}
\hline
\textbf{Filozofowie} & \textbf{Czas [s]} & \textbf{Posiłki} & \textbf{Throughput} & \textbf{Fairness [\%]} \\
\hline
3 & 30 & 152 & 4.97 & 66.15 \\
3 & 60 & 301 & 4.98 & 66.67 \\
3 & 180 & 900 & 4.99 & 64.38 \\
3 & 600 & 2996 & 4.99 & 65.28 \\
3 & 1200 & 5990 & 4.99 & 67.36 \\
\hline
5 & 30 & 303 & 9.90 & 80.30 \\
5 & 60 & 601 & 9.93 & 76.69 \\
5 & 180 & 1799 & 9.96 & 81.96 \\
5 & 600 & 5987 & 9.97 & 78.69 \\
5 & 1200 & 11969 & 9.97 & 77.70 \\
\hline
16 & 30 & 1200 & 39.54 & 100.00 \\
16 & 60 & 2243 & 37.22 & 94.44 \\
16 & 180 & 7106 & 39.40 & 98.44 \\
16 & 600 & 23755 & 39.56 & 99.53 \\
16 & 1200 & 47563 & 39.62 & 99.80 \\
\hline
32 & 30 & 2162 & 71.11 & 81.33 \\
32 & 60 & 4548 & 75.43 & 92.62 \\
32 & 180 & 14048 & 77.87 & 96.42 \\
32 & 600 & 44849 & 74.68 & 98.72 \\
32 & 1200 & 94816 & 78.97 & 100.00 \\
\hline
\end{tabular}
\end{table}

\textbf{Kluczowe obserwacje:}
\begin{itemize}
    \item \textbf{Zero deadlocków} -- strategia skutecznie zapobiega zakleszczeniom
    \item \textbf{Skalowalność:} Przepustowość rośnie liniowo z liczbą filozofów
    \item \textbf{Stabilność:} Throughput pozostaje stały niezależnie od czasu symulacji
    \item \textbf{Fairness coefficient:} Rośnie z czasem symulacji, osiągając 100\% dla 32 filozofów w 1200s
    \item Najlepsza sprawiedliwość przy większej liczbie filozofów
\end{itemize}

\subsection{Strategia Asymmetric -- przerwanie symetrii}

Strategia asymmetryczna (parzyści filozofowie: lewy→prawy, nieparzyści: prawy→lewy) również skutecznie zapobiega deadlockom.

\begin{table}[h]
\centering
\caption{Wyniki strategii Asymmetric -- pełne wyniki}
\small
\begin{tabular}{|c|c|c|c|c|}
\hline
\textbf{Filozofowie} & \textbf{Czas [s]} & \textbf{Posiłki} & \textbf{Throughput} & \textbf{Fairness [\%]} \\
\hline
3 & 30 & 152 & 4.97 & 61.76 \\
3 & 60 & 301 & 4.98 & 65.38 \\
3 & 180 & 900 & 4.98 & 62.66 \\
3 & 600 & 2996 & 4.99 & 64.93 \\
3 & 1200 & 5990 & 4.99 & 65.28 \\
\hline
5 & 30 & 302 & 9.87 & 87.88 \\
5 & 60 & 601 & 9.93 & 81.16 \\
5 & 180 & 1797 & 9.96 & 80.15 \\
5 & 600 & 5987 & 9.97 & 82.24 \\
5 & 1200 & 11873 & 9.89 & 88.88 \\
\hline
16 & 30 & 1193 & 39.36 & 98.67 \\
16 & 60 & 2370 & 39.30 & 98.66 \\
16 & 180 & 7134 & 39.57 & 99.55 \\
16 & 600 & 23832 & 39.70 & 99.93 \\
16 & 1200 & 47702 & 39.73 & 99.93 \\
\hline
32 & 30 & 2399 & 79.09 & 98.67 \\
32 & 60 & 4769 & 78.95 & 99.33 \\
32 & 180 & 14286 & 79.20 & 99.55 \\
32 & 600 & 47649 & 79.36 & 99.93 \\
32 & 1200 & 95409 & 79.45 & 99.97 \\
\hline
\end{tabular}
\end{table}

\textbf{Kluczowe obserwacje:}
\begin{itemize}
    \item \textbf{Zero deadlocków} -- równie skuteczna jak hierarchy
    \item \textbf{Wyższa przepustowość:} Przy 32 filozofach osiąga 79.45 meals/s (vs 78.97 hierarchy)
    \item \textbf{Doskonała sprawiedliwość:} Fairness coefficient 98-100\% dla większych grup
    \item \textbf{Bardzo niska wariancja:} Standard deviation < 1 dla 16 i 32 filozofów
    \item Lepsze wykorzystanie zasobów dzięki asymetrycznemu dostępowi
\end{itemize}

\subsection{Porównanie strategii}

\begin{table}[h]
\centering
\caption{Porównanie strategii dla 32 filozofów, 1200s}
\begin{tabular}{|l|c|c|c|}
\hline
\textbf{Metryka} & \textbf{Deadlock} & \textbf{Hierarchy} & \textbf{Asymmetric} \\
\hline
Całkowite posiłki & 0 & 94816 & 95409 \\
Throughput [meals/s] & 0 & 78.97 & 79.45 \\
Deadlock detected & TAK & NIE & NIE \\
Fairness coefficient & - & 100\% & 99.97\% \\
Std deviation & 0 & 0.0 & 0.5 \\
\hline
\end{tabular}
\end{table}

\subsection{Wpływ liczby filozofów na wydajność}

\textbf{Dla strategii hierarchy i asymmetric:}

\begin{enumerate}
    \item \textbf{3 filozofy:} $\approx$ 5 meals/s
    \begin{itemize}
        \item Ograniczona równoległość (max 1 para jedząca jednocześnie)
        \item Wysoka sprawiedliwość utrudniona (66-67\%)
    \end{itemize}
    
    \item \textbf{5 filozofów:} $\approx$ 10 meals/s (2x wzrost)
    \begin{itemize}
        \item Max 2 pary mogą jeść jednocześnie
        \item Lepsza sprawiedliwość (77-89\%)
    \end{itemize}
    
    \item \textbf{16 filozofów:} $\approx$ 40 meals/s (4x wzrost)
    \begin{itemize}
        \item Wysokie wykorzystanie 16 wątków sprzętowych
        \item Doskonała sprawiedliwość (98-100\%)
    \end{itemize}
    
    \item \textbf{32 filozofy:} $\approx$ 79 meals/s (8x wzrost)
    \begin{itemize}
        \item Przekroczenie liczby wątków sprzętowych (32 > 16)
        \item Overhead context switching, ale wciąż wysoka wydajność
        \item Najlepsza sprawiedliwość w długich testach
    \end{itemize}
\end{enumerate}

\subsection{Analiza sprawiedliwości (Fairness)}

\textbf{Współczynnik sprawiedliwości} = (min\_meals / max\_meals) × 100\%

\begin{itemize}
    \item \textbf{Hierarchy:} 64-100\%, rośnie z liczbą filozofów i czasem symulacji
    \item \textbf{Asymmetric:} 61-100\%, również rośnie z parametrami
    \item W testach 1200s obie strategie osiągają $\geq$ 99\% sprawiedliwości
    \item Dla 32 filozofów w długim teście: niemal idealna równość posiłków
\end{itemize}

\section{Wnioski}

\subsection{Podsumowanie wyników}

Przeprowadzone eksperymenty potwierdzają teoretyczne przewidywania dotyczące problemu ucztujących filozofów:

\begin{enumerate}
    \item \textbf{Deadlock jest realnym zagrożeniem:} Naiwne podejście prowadzi do natychmiastowego zakleszczenia niezależnie od liczby filozofów czy czasu symulacji.
    
    \item \textbf{Hierarchia zasobów działa:} Rozwiązanie Dijkstry całkowicie eliminuje deadlock, zapewniając stabilną wydajność i wysoką sprawiedliwość.
    
    \item \textbf{Asymetria jest efektywna:} Przerwanie symetrii działań filozofów nie tylko zapobiega deadlockowi, ale również oferuje nieco wyższą przepustowość niż hierarchy.
    
    \item \textbf{Wielordzeniowość ma znaczenie:} Na procesorze 8-rdzeniowym (16 wątków) widoczny jest liniowy wzrost wydajności do 16 filozofów. Przy 32 filozofach pojawia się overhead context switching, ale system pozostaje efektywny.
\end{enumerate}

\subsection{Porównanie strategii}

\textbf{Hierarchy (Dijkstra):}
\begin{itemize}
    \item[+] Matematycznie dowiedlna poprawność
    \item[+] Deterministyczna kolejność dostępu do zasobów
    \item[+] Doskonała sprawiedliwość w długich symulacjach (100\%)
    \item[--] Nieco niższa przepustowość niż asymmetric przy dużej liczbie filozofów
\end{itemize}

\textbf{Asymmetric (Parzyści/Nieparzyści):}
\begin{itemize}
    \item[+] Najwyższa przepustowość (79.45 meals/s dla 32 filozofów)
    \item[+] Prostsza implementacja niż hierarchy
    \item[+] Bardzo dobra sprawiedliwość (99.97\%)
    \item[+] Niższa wariancja w rozdziale posiłków
    \item[--] Mniej intuicyjna niż hierarchy dla małych grup
\end{itemize}

\subsection{Wnioski końcowe}

\begin{enumerate}
    \item Wybór strategii synchronizacji ma \textbf{krytyczne znaczenie} dla wydajności i bezpieczeństwa systemu wielowątkowego.
    
    \item W systemach produkcyjnych należy \textbf{zawsze} stosować sprawdzone mechanizmy zapobiegania deadlockom (hierarchy, asymmetry, timeouts).
    
    \item Przy projektowaniu systemów współbieżnych konieczne jest uwzględnienie:
    \begin{itemize}
        \item Liczby dostępnych wątków sprzętowych
        \item Oczekiwanego poziomu równoległości
        \item Wymagań co do sprawiedliwości dostępu
    \end{itemize}
    
    \item \textbf{Testing w warunkach realnych} (prawdziwa wielordzeniowość) ujawnia problemy niewidoczne w symulacjach jednordzeniowych.
    
    \item Dla systemów wymagających najwyższej wydajności, strategia \textbf{asymmetric} oferuje najlepsze wyniki przy zachowaniu bezpieczeństwa.
\end{enumerate}

\section{Bibliografia}

\begin{enumerate}
    \item Dijkstra, E. W. (1971). \textit{Hierarchical ordering of sequential processes}. Acta Informatica.
    \item Tanenbaum, A. S., \& Bos, H. (2014). \textit{Modern Operating Systems} (4th ed.). Pearson.
    \item Silberschatz, A., Galvin, P. B., \& Gagne, G. (2018). \textit{Operating System Concepts} (10th ed.). Wiley.
\end{enumerate}

\end{document}
